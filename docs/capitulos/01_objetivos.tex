\chapter{Objetivos}

El objetivo general del proyecto es enfrentarse a un problema de Machine Learning real con una metodología ágil para producir y validar modelos predictivos e intentar obtener los mejores resultados posibles y que éstos sean reproducibles. De forma más específica, los objetivos serían los siguientes:

\begin{itemize}
	\item Llevar un control sobre los experimentos realizados así como de los artefactos generados con cada experimento.
	\item Obtener los mejores resultados posibles mediante el uso de los hiperparámetros óptimos en el entrenamiento de cada modelo.
	\item Controlar la validación de los modelos mediante tests, es decir, que las entradas y las salidas de los mismos sea la esperada y que se optimicen correctamente.
	\item Despliegue de modelos de forma ágil, sin que el desarrollador tenga que gastar mucho tiempo en ello y pueda concentrarse en escribir/mejorar el código.
	\item Todo el código de este proyecto será liberado para que cualquier persona pueda consultarlo con fines académicos y/o colaborar mediante algún cambio o añadiendo alguna \textit{feature}. Obviamente tendrá su licencia para evitar plagios o algún otro uso indebido.
	\item Por último y no menos importante, este trabajo y sus resultados deben ser completamente reproducibles. Es decir, que puedan ser ejecutados en otro computador distinto con facilidad y que se obtengan los mismos resultados.
\end{itemize}