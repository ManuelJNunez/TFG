\chapter*{}
%\thispagestyle{empty}
%\cleardoublepage

%\thispagestyle{empty}



\cleardoublepage
\thispagestyle{empty}

\begin{center}
{\large\bfseries Desarrollo de Modelos de Machine Learning: Aplicando metodologías ágiles}\\
\end{center}
\begin{center}
Manuel Jesús Núñez Ruiz\\
\end{center}

%\vspace{0.7cm}
\noindent{\textbf{Palabras clave}: Machine Learning, Deep Learning, DevOps, MLOps, Integración Continua, Despliegue Continuo, Redes Convolucionales, Autoencoders.}\\

\vspace{0.7cm}
\noindent{\textbf{Resumen}}\\

En este proyecto se pretende diseñar e implementar un flujo de trabajo ágil a la vez que se trabaja sobre un conjunto de datos del campo de la física de partículas.\\

A la hora de diseñar dicho flujo de trabajo ágil se ha tenido en cuenta la automatización de los \textit{tests} para comprobar que los cambios introducidos son correctos y el uso de tecnologías libres para que dicho flujo de trabajo sea reproducible en cualquier máquina por cualquier persona u organización. Por tanto, toda la infraestructura virtual levantada y usada para la realización de este proyecto debe ser completamente reproducible.\\

Además de todo lo anterior, en este nuevo flujo de trabajo se deben de recoger automáticamente los hiperparámetros con los que el modelo fue entrenado, las métricas resultantes que miden la precisión predictiva del modelo tras el ajuste y los artefactos generados. Entiéndase por estos últimos los objetos o ficheros generados por el entrenamiento del modelo que pueden ser imágenes, modelos salvados, binarios ejecutables, etc.\\

Por último, también señalar la importancia del uso de herramientas para asegurar la calidad y claridad del código fuente.
\cleardoublepage


\thispagestyle{empty}


\begin{center}
{\large\bfseries Machine Learning Models Development: Applying agile methodologies}\\
\end{center}
\begin{center}
Manuel Jesús, Núñez Ruiz\\
\end{center}

%\vspace{0.7cm}
\noindent{\textbf{Keywords}: Machine Learning, Deep Learning, DevOps, MLOps, Continuous Integration, Continuous Deployment, Convolutional Networks, Autoencoders.}\\

\vspace{0.7cm}
\noindent{\textbf{Abstract}}\\

This project aims to design and implement an agile workflow and at the same time to work with a dataset from the field of particle physics.\\

When designing the above-mentioned agile workflow, test automatization has been taken into account to check whether the introduced changes are correct and the use of open-source technology so that this workflow can be reproduced in any machine by any user or organization. Therefore, all the virtual infrastructure that have been launched and used for the realization of this project should be completely reproducible.\\

In addition to all the above, in this new workflow, it will be necessary to automatically collect the hyperparameters with which the model was trained, the resulting metrics used to measure the model's predictive accuracy after being fitted, and the generated artifacts. This last concept refers to all the objects or files generated during the training of the model that can be images, saved models, executables, etc.\\

Finally, I would like to point out the importance of the use of some tools to ensure the quality and the clarity of the source code.

\chapter*{}
\thispagestyle{empty}

\noindent\rule[-1ex]{\textwidth}{2pt}\\[4.5ex]

Yo, \textbf{Manuel Jesús Núñez Ruiz}, alumno de la titulación Grado en Ingeniería Informática de la \textbf{Escuela Técnica Superior
de Ingenierías Informática y de Telecomunicación de la Universidad de Granada}, con DNI 31027610N, autorizo la
ubicación de la siguiente copia de mi Trabajo Fin de Grado en la biblioteca del centro para que pueda ser
consultada por las personas que lo deseen.

\vspace{6cm}

\noindent Fdo: Manuel Jesús Núñez Ruiz.

\vspace{2cm}

\begin{flushright}
Granada a 28 de Junio de 2021.
\end{flushright}


\chapter*{}
\thispagestyle{empty}

\noindent\rule[-1ex]{\textwidth}{2pt}\\[4.5ex]

D. \textbf{Alberto Guillén Perales}, Profesor del Departamento de Arquitectura y Tecnologı́a de Computadores de la Universidad de Granada.


\vspace{0.5cm}

\textbf{Informa:}

\vspace{0.5cm}

Que el presente trabajo, titulado \textit{\textbf{Desarrollo de Modelos de Machine Learning, Aplicando metodologías ágiles}},
ha sido realizado bajo su supervisión por \textbf{Manuel Jesús Núñez Ruiz}, y autorizamos la defensa de dicho trabajo ante el tribunal
que corresponda.

\vspace{0.5cm}

Y para que conste, expiden y firman el presente informe en Granada a 9 de Julio de 2021.

\vspace{1cm}

\textbf{El tutor:}

\vspace{5cm}

\noindent \textbf{Alberto Guillén Perales}

\chapter*{Agradecimientos}
\thispagestyle{empty}

       \vspace{1cm}


Antes que nada me gustaría agradecer a mi familia los cuales siempre me han apoyado en todo y me han dado la oportunidad de poder estudiar y llegar a la realización de este trabajo.\\

También me gustaría agradecer a mis amigos cercanos que siempre me han brindado su apoyo en todo lo que he hecho y me han ayudado mucho a despejarme cuando estaba abrumado.\\

A mis compañeros de informática, que aunque casi todos son mis amigos cercanos, siempre nos hemos estado dando ánimos con la difícil situación que tuvimos con el COVID-19 y hemos salido adelante sin problemas ayudándonos los unos a los otros en todo lo posible y compartiendo muchos buenos momentos en la Posada (espero que Paco tenga en cuenta la publicidad gratuita) y en nuestras quedadas (ya sean virtuales o en persona).\\

A mis compañeros de \textit{Real-Time Innovations} por haber hecho que mi primera experiencia laboral sea difícil de superar y de los que he tenido la oportunidad de aprender un montón y siendo siempre tan simpáticos y humildes.\\

A las personas del \textit{Southern Wide-field Gamma-ray Observatory} (\textit{SWGO}) y del \textit{Laboratório de Instrumentação e Física Experimental de Partículas} (\textit{LIP}), más concretamente a Mario Pimenta, Ruben Conceição y Bernardo Tomé, por proveerme del conjunto de datos obtenido de sus simulaciones para poder realizar este trabajo.\\

Y por último pero no menos importate, a mi tutor durante la realización de este trabajo Alberto por haberme puesto en contacto con las personas anteriores y haber hecho este trabajo tan interesante posible. Y por supuesto, por ser una gran persona además de un gran profesor que lleva a cabo su docencia con ganas de enseñar y aprender, pues en esta disciplina nunca se para de aprender cosas nuevas e interesantes.
