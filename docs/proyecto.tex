\documentclass[a4paper,11pt]{book}
%\documentclass[a4paper,twoside,11pt,titlepage]{book}
\usepackage{listings}
\usepackage[utf8]{inputenc}
\usepackage[spanish]{babel}

\usepackage{float}
\usepackage{csquotes}

% Matemáticas y algoritmos
\usepackage{amsmath}
\usepackage{amsfonts}
\usepackage[linesnumbered,ruled,vlined]{algorithm2e}

% Comandos matemáticos
\DeclareMathOperator*{\argmin}{arg\,min}

% nonl command
\let\oldnl\nl
\newcommand{\nonl}{\renewcommand{\nl}{\let\nl\oldnl}}

%\usepackage[style=list, number=none]{glossary} %
%\usepackage{titlesec}
%\usepackage{pailatino}

\decimalpoint
\usepackage{dcolumn}
\newcolumntype{.}{D{.}{\esperiod}{-1}}
\makeatletter
\addto\shorthandsspanish{\let\esperiod\es@period@code}
\makeatother


%\usepackage[chapter]{algorithm}
\RequirePackage{verbatim}
%\RequirePackage[Glenn]{fncychap}
\usepackage{fancyhdr}
\usepackage{graphicx}
\usepackage{afterpage}

\usepackage{longtable}

\usepackage[pdfborder={000}]{hyperref} %referencia

% ********************************************************************
% Re-usable information
% ********************************************************************
\newcommand{\myTitle}{Desarrollo de modelos de Machine Learning aplicando MLOps\xspace}
\newcommand{\myDegree}{Grado en Ingeniería Informática\xspace}
\newcommand{\myName}{Manuel Jesús Núñez Ruiz (alumno)\xspace}
\newcommand{\myProf}{Alberto Guillén Perales(tutor)\xspace}
%\newcommand{\mySupervisor}{Put name here\xspace}
\newcommand{\myFaculty}{Escuela Técnica Superior de Ingenierías Informática y de
Telecomunicación\xspace}
\newcommand{\myFacultyShort}{E.T.S. de Ingenierías Informática y de
Telecomunicación\xspace}
\newcommand{\myDepartment}{Departamento de Arquitectura y Tecnología de Computadores\xspace}
\newcommand{\myUni}{\protect{Universidad de Granada}\xspace}
\newcommand{\myLocation}{Granada\xspace}
\newcommand{\myTime}{\today\xspace}
\newcommand{\myVersion}{Version 0.1\xspace}


\hypersetup{
pdfauthor = {\myName manueljesusnunezruiz@gmail.com},
pdftitle = {\myTitle},
pdfsubject = {},
pdfkeywords = {palabra_clave1, palabra_clave2, palabra_clave3, ...},
pdfcreator = {pdflatex},
pdfproducer = {pdflatex}
}

%\hyphenation{}


%\usepackage{doxygen/doxygen}
%\usepackage{pdfpages}
\usepackage{url}
\usepackage{colortbl,longtable}
\usepackage[stable]{footmisc}
\usepackage{index}

%\makeindex
%\usepackage[style=long, cols=2,border=plain,toc=true,number=none]{glossary}
% \makeglossary

% Definición de comandos que me son tiles:
%\renewcommand{\indexname}{Índice alfabético}
%\renewcommand{\glossaryname}{Glosario}

\pagestyle{fancy}
\fancyhf{}
\fancyhead[LO]{\leftmark}
\fancyhead[RE]{\rightmark}
\fancyhead[RO,LE]{\textbf{\thepage}}
\renewcommand{\chaptermark}[1]{\markboth{\textbf{#1}}{}}
\renewcommand{\sectionmark}[1]{\markright{\textbf{\thesection. #1}}}

\setlength{\headheight}{1.5\headheight}

\newcommand{\HRule}{\rule{\linewidth}{0.5mm}}
%Definimos los tipos teorema, ejemplo y definición podremos usar estos tipos
%simplemente poniendo \begin{teorema} \end{teorema} ...
\newtheorem{teorema}{Teorema}[chapter]
\newtheorem{ejemplo}{Ejemplo}[chapter]
\newtheorem{definicion}{Definición}[chapter]

\definecolor{gray97}{gray}{.97}
\definecolor{gray75}{gray}{.75}
\definecolor{gray45}{gray}{.45}
\definecolor{gray30}{gray}{.94}

\lstset{ frame=Ltb,
     framerule=0.5pt,
     aboveskip=0.5cm,
     framextopmargin=3pt,
     framexbottommargin=3pt,
     framexleftmargin=0.1cm,
     framesep=0pt,
     rulesep=.4pt,
     backgroundcolor=\color{gray97},
     rulesepcolor=\color{black},
     %
     stringstyle=\ttfamily,
     showstringspaces = false,
     basicstyle=\scriptsize\ttfamily,
     commentstyle=\color{gray45},
     keywordstyle=\bfseries,
     %
     numbers=left,
     numbersep=6pt,
     numberstyle=\tiny,
     numberfirstline = false,
     breaklines=true,
   }
 
% minimizar fragmentado de listados
\lstnewenvironment{listing}[1][]
   {\lstset{#1}\pagebreak[0]}{\pagebreak[0]}

\lstdefinestyle{CodigoC}
   {
	basicstyle=\scriptsize,
	frame=single,
	language=C,
	numbers=left
   }
\lstdefinestyle{CodigoC++}
   {
	basicstyle=\small,
	frame=single,
	backgroundcolor=\color{gray30},
	language=C++,
	numbers=left
   }

 
\lstdefinestyle{Consola}
   {basicstyle=\scriptsize\bf\ttfamily,
    backgroundcolor=\color{gray30},
    frame=single,
    numbers=none
   }


\newcommand{\bigrule}{\titlerule[0.5mm]}


%Para conseguir que en las páginas en blanco no ponga cabecerass
\makeatletter
\def\clearpage{%
  \ifvmode
    \ifnum \@dbltopnum =\m@ne
      \ifdim \pagetotal <\topskip
        \hbox{}
      \fi
    \fi
  \fi
  \newpage
  \thispagestyle{empty}
  \write\m@ne{}
  \vbox{}
  \penalty -\@Mi
}
\makeatother

\usepackage{pdfpages}
\begin{document}
\begin{titlepage}
 
 
\newlength{\centeroffset}
\setlength{\centeroffset}{-0.5\oddsidemargin}
\addtolength{\centeroffset}{0.5\evensidemargin}
\thispagestyle{empty}

\noindent\hspace*{\centeroffset}\begin{minipage}{\textwidth}

\centering
\includegraphics[width=0.9\textwidth]{imagenes/logo_ugr.jpg}\\[1.4cm]

\textsc{ \Large TRABAJO FIN DE GRADO\\[0.2cm]}
\textsc{ INGENIERÍA INFORMÁTICA}\\[1cm]
% Upper part of the page
% 
% Title
{\Huge\bfseries Desarrollo de modelos de Machine Learning\\
}
\noindent\rule[-1ex]{\textwidth}{3pt}\\[3.5ex]
{\large\bfseries Aplicando metodologías ágiles}
\end{minipage}

\vspace{2.5cm}
\noindent\hspace*{\centeroffset}\begin{minipage}{\textwidth}
\centering

\textbf{Autor}\\ {Manuel Jesús Núñez Ruiz (alumno)}\\[2.5ex]
\textbf{Directores}\\
{Alberto Guillén Perales (tutor)}\\[2cm]
\includegraphics[width=0.3\textwidth]{imagenes/etsiit_logo.png}\\[0.1cm]
\textsc{Escuela Técnica Superior de Ingenierías Informática y de Telecomunicación}\\
\textsc{---}\\
Granada, Julio de 2021
\end{minipage}
%\addtolength{\textwidth}{\centeroffset}
%\vspace{\stretch{2}}
\end{titlepage}



%\chapter*{}
%\thispagestyle{empty}
%\cleardoublepage

%\thispagestyle{empty}

\cleardoublepage
\thispagestyle{empty}

\begin{center}
{\large\bfseries Título del Proyecto: Subtítulo del proyecto}\\
\end{center}
\begin{center}
Nombre Apellido1 Apellido2 (alumno)\\
\end{center}

%\vspace{0.7cm}
\noindent{\textbf{Palabras clave}: palabra\_clave1, palabra\_clave2, palabra\_clave3, ......}\\

\vspace{0.7cm}
\noindent{\textbf{Resumen}}\\

Poner aquí el resumen.
%\cleardoublepage
\newpage


\thispagestyle{empty}


\begin{center}
{\large\bfseries Project Title: Project Subtitle}\\
\end{center}
\begin{center}
First name, Family name (student)\\
\end{center}

%\vspace{0.7cm}
\noindent{\textbf{Keywords}: Keyword1, Keyword2, Keyword3, ....}\\

\vspace{0.7cm}
\noindent{\textbf{Abstract}}\\

Write here the abstract in English.

\chapter*{}
\thispagestyle{empty}

\noindent\rule[-1ex]{\textwidth}{2pt}\\[4.5ex]

Yo, \textbf{Manuel Jesús Núñez Ruiz}, alumno de la titulación TITULACIÓN de la \textbf{Escuela Técnica Superior
de Ingenierías Informática y de Telecomunicación de la Universidad de Granada}, con DNI 31027610N, autorizo la
ubicación de la siguiente copia de mi Trabajo Fin de Grado en la biblioteca del centro para que pueda ser
consultada por las personas que lo deseen.

\vspace{6cm}

\noindent Fdo: Manuel Jesús Núñez Ruiz

\vspace{2cm}

\begin{flushright}
Granada a X de Junio de 2021.
\end{flushright}


\chapter*{}
\thispagestyle{empty}

\noindent\rule[-1ex]{\textwidth}{2pt}\\[4.5ex]

D. \textbf{Alberto Guillén Perales (tutor)}, Profesor del Área de XXXX del Departamento Arquitectura y Tecnología de Computadores de la Universidad de Granada.



\vspace{0.5cm}

\textbf{Informan:}

\vspace{0.5cm}

Que el presente trabajo, titulado \textit{\textbf{Desarrollo de modelos de Machine Learning, Aplicando metodologías ágiles y las mejores prácticas}},
ha sido realizado bajo su supervisión por \textbf{Manuel Jesús Núñez Ruiz (alumno)}, y autorizamos la defensa de dicho trabajo ante el tribunal
que corresponda.

\vspace{0.5cm}

Y para que conste, expiden y firman el presente informe en Granada a X de Junio de 2021.

\vspace{1cm}

\textbf{El director:}

\vspace{5cm}

\noindent \textbf{Alberto Guillén Perales (tutor)}

\chapter*{Agradecimientos}
\thispagestyle{empty}

       \vspace{1cm}


Poner aquí agradecimientos...


%\frontmatter
\tableofcontents
%\listoffigures
%\listoftables
%
%\mainmatter
%\setlength{\parskip}{5pt}

\chapter{Objetivos}

El objetivo general del proyecto es enfrentarse a un problema de Machine Learning real con una metodología ágil para producir y validar modelos predictivos e intentar obtener los mejores resultados posibles y que éstos sean reproducibles. De forma más específica, los objetivos serían los siguientes:

\begin{itemize}
	\item Llevar un control sobre los experimentos realizados así como de los artefactos generados con cada experimento.
	\item Obtener los mejores resultados posibles mediante el uso de los hiperparámetros óptimos en el entrenamiento de cada modelo.
	\item Controlar la validación de los modelos mediante tests, es decir, que las entradas y las salidas de los mismos sea la esperada y que se optimicen correctamente.
	\item Despliegue de modelos de forma ágil, sin que el desarrollador tenga que gastar mucho tiempo en ello y pueda concentrarse en escribir/mejorar el código.
	\item Todo el código de este proyecto será liberado para que cualquier persona pueda consultarlo con fines académicos y/o colaborar mediante algún cambio o añadiendo alguna \textit{feature}. Obviamente tendrá su licencia para evitar plagios o algún otro uso indebido.
	\item Por último y no menos importante, este trabajo y sus resultados deben ser completamente reproducibles. Es decir, que puedan ser ejecutados en otro computador distinto con facilidad y que se obtengan los mismos resultados.
\end{itemize}

\chapter{Introducción}

\section{¿Qué es DevOps?}

\enquote{El término \textit{DevOps}, que es una combinación de los términos ingleses \textit{development} (desarrollo) y \textit{operations} (operaciones), designa la unión de personas, procesos y tecnología para ofrecer valor a los clientes de forma constante} \cite{devops}. Gracias a esta práctica y a sus herramientas, los roles que trabajan en el desarrollo del software pueden trabajar juntos para producir productos mejores y más confiables, además de hacer que mejore el rendimiento de los equipos y creen productos de calidad en menos tiempo.

\subsection{Prácticas DevOps}

La mayoría de las prácticas de \textit{DevOps} están diseñadas para agilizar, automatizar y mejorar las fases del desarrollo del software, reduciendo el riesgo de errores humanos y mejorando la velocidad.

\begin{itemize}
	\item \textbf{Integración y Despliegue Continuos (CI/CD).} Mediante la práctica de Integración Continua, cada vez que se fusionan los cambios en la rama de código principal (cada vez que se realiza un \enquote{commit}), se realizan pruebas automáticas para garantizar la estabilidad de la rama principal \cite{devops}.\\
	
	El Despliegue Continuo es la automatización de la puesta en producción de los nuevos cambios y también se ejecuta al fusionar nuevos cambios en la rama principal. De esta forma, es mucho más rápido el proceso del despliegue al no tener que hacerlo manualmente y nos permite realizar implementaciones más frecuentes \cite{devops}.
	
	\item \textbf{Control de versiones.} \enquote{Es la práctica que nos permite administrar el código por versiones y del historial de cambios para facilitar la revisión y la recuperación del código} \cite{devops}. El sistema que nos brinda esta utilidad es \textit{git} y el código suele estar publicado en un repositorio remoto (que puede ser privado o público) usando este mismo sistema para que todo el equipo tenga acceso al control de versiones.
	
	\item \textbf{Infraestructura como Código.} \enquote{Permite definir las topologías y los recusos de un modo descriptivo que permite a los equipos administrar esos recursos igual que lo harían con el código} \cite{devops}. Además, al estar definida como código, se puede administrar sus versiones con el sistema de control de versiones. En adición, hace que la creación de la infraestructura sea repetible y ayuda a automatizar el Despliegue continuo.
	
	\item \textbf{Administración de configuración.} Podemos usarla junto con la Infraestructura como Código para automatizar la definición y configuración de sistemas \cite{devops}. Es muy común guardar la configuración de un sistema en un fichero con un formato determinado (si no contiene información confidencial como contraseñas), como variable de entorno en una única máquina o usar un mecanismo de configuración distribuida si utilizamos un cluster (como por ejemplo etcd \cite{etcd}).
	
	\item \textbf{Monitorización.} Consiste en visualizar en tiempo real el rendimiento y el estado de la pila de aplicaciones, desde la infraestructura subyacente hasta el software de niveles superiores con el fin de mitigar los errores producidos durante la ejecución de la misma o ver si es necesario escalarla \cite{devops}.
\end{itemize}

\subsection{DevOps y contenedores}

\enquote{Los contenedores linux son una tecnología que nos permite empaquetar y aislar aplicaciones junto con su entorno de ejecución (todos los archivos necesarios para que funcione)} \cite{containers}. Esto nos permite que la aplicación se pueda ejecutar en cualquier máquina/arquitectura que ejecute Linux sin problemas (solo se tiene que tener instalado algún \textit{container engine}) y con mucha facilidad, pues los contenedores son portables. Para esto, existe en internet un gran número de repositorios de contenedores, que  nos permite subirlos de manera gratuita y descargarlo en otra máquina. Gracias a esta tecnología, podemos facilitar mucho el flujo de CI/CD al poder ejecutar la aplicación y sus tests en un contenedor.\newline

\section{Motivación para usar MLOps}

Hoy en día, en los proyectos de Machine Learning, el código dedicado a definir los modelos predictivos representa una fracción muy pequeña comparada con otros componentes del flujo de trabajo. Para poder operar con modelos, los científicos de datos deben de trabajar junto con otras personas que se dedican a eso mismo, operaciones (encargados de desplegar, testear el sistema, aprovisionar la infraestructura necesaria, etc). Esto representa un desafío en la organización en términos de comunicación, colaboración y coordinación. El objetivo de MLOps es enfrentarse a dicho desafío estableciendo buenas prácticas para el desarrollo. Además, nos aporta de la velocidad y agilidad necesaria en el mundo digital actual.

\begin{figure}[h]
	\includegraphics[scale=1]{imagenes/01_Introduccion/mlcodefraction.jpg}
	\centering
	\caption{Como se muestra en la figura, la fracción de código en los sistemas de ML reales es bastante pequeña. En cambio la fracción de infraestructura es mucho mayor \cite{NIPS2015_86df7dcf}.}
\end{figure}

En este caso, adoptar las mejores prácticas de \textbf{DevOps} podría ser posible, pues el Machine Learning tiene mucho en común con la ingeniería del software. La diferencia es que en Machine Learning hay que llevar un registro de modelos y sus hiperparámetros, con el fin de que el experimento se pueda replicar con la mayor facilidad en otro sistema (esto recibe el nombre de reproducibilidad) \cite{whymlops}.\newline

En DevOps y MLOps se debe de usar un sistema de control de versiones (git) para poder gestionar los cambios en el código. Adicionalmente, en MLOps necesitamos versionar los modelos y sus hiperparámetros, pero gracias a la herramienta DVC (su uso está explicado en un capítulo posterior) esto ya no es problema. Además de todo lo anterior, el código debe de estar subido a un repositorio remoto compartido (como puede ser GitHub, GitLab, Bitbucket, etc) para poder trabajar cómodamente en equipo. En este repositorio debemos tener el control de que en cada cambio incremental todo lo anterior y lo nuevo siga funcionando, por lo tanto es necesario el uso de lo que se llama Integración Continua. Este proceso de Integración Continua lo que hace es ejecutar los tests que hemos escrito sobre el código implementado para comprobar que efectivamente funciona como esperábamos. En cuanto a MLOps, también es necesario testear que los modelos se comporten como esperamos. Esto nos permite desarrollar software de mayor calidad y con una altísima frecuencia de releases.\newline

Además de lo anterior, también se puede configurar un Despliegue Continuo, así si nuestro código pasa con éxito los tests de Integración Continua, podriamos desplegarlo en cuestión de segundos o unos pocos minutos mediante otras herramientas/plataformas que nos permitan acelerar dicho despliegue usando Infraestructura como Código. Para poder integrar la Integración Continua y el Despliegue Continuo (a partir de ahora lo llamaremos CI/CD) nos hace falta una plataforma en la nube que nos lo permita. Todo esto está desarrollado en los próximos capítulos de este documento.

\section{Descripción del problema}

La Tierra está siendo \enquote{bombardeada} constantemente con rayos cósmicos que provienen, como su nombre indica, del universo. Estos rayos se componen de partículas que viajan a la velocidad de la luz, que al entrar en contacto con la atmósfera producen una cascada atmosférica extensa. Dicha cascada emite una luz debido a la radiación gamma llamada \enquote{Luz de Cherenkov}. A la misma vez, el rayo cósmico se fragmenta formando hadrones. Es decir, estos rayos se descomponen en una componente hadrónica y otra componente electromagnética.\newline

\enquote{La detección de de rayos gamma de muy alta energía es esencial para investigar las fuentes de la radiación entrante producida por alguno de los fenómenos más extremos que suceden en el universo, por ejemplo estrellas de neutrones y agujeros negros supermasivos} \cite{gonzalez2021tackling}.

\begin{figure}[H]
	\includegraphics[scale=.3]{imagenes/01_Introduccion/EAS.jpg}
	\centering
	\label{fig:wcd}
	\caption{Así sería una Cascada Atmosférica Extensa incidiendo sobre un Water Cherenkov Detector (WCD). Fuente: \href{https://revolucion.news/cienciario.mx/los-rayos-cosmicos-de-muy-alta-energia-vienen-de-fuera-de-la-via-lactea/}{revolucion.news}.}
\end{figure}

Para poder capturar información sobre dichos rayos cósmicos, se usan WCD (se puede ver la imagen de uno en la Figura \ref{fig:wcd}) que son tanques de agua ultrapura que contienen fotomultiplicadores distribuidos simétricamente dentro del tanque que se utilizan para captar la señal del rayo que incide sobre el mismo. En función de la profundidad del agua del WCD, se emitirá más luz o menos, siendo directamente proporcionales.\newline

El problema que se pretende solucionar en este proyecto, es diseñar un modelo que sea capaz de diferenciar un rayo que sea solo radiación gamma electromagnética de uno que tenga una partícula denominada \enquote{muón}. Esta partícula se genera por la interacción a alta energía de los hadrones y nos permiten discriminar bastante bien entre la radiación gamma y los hadrones \cite{gonzalez2021tackling}.

\begin{figure}[H]
	\includegraphics[scale=.1]{imagenes/01_Introduccion/wcd.png}
	\centering
	\caption{Diseño de un WCD \cite{gonzalez2021tackling}.}
\end{figure}


\chapter{Estado del arte}

\section{Ingeniería del Software}

En esta sección voy a describir las herramientas que se usarán en este proyecto así como la justificación del uso de cada una. Todas están enfocadas para lograr las mejores prácticas de la metodología DevOps.

\subsection{Lenguaje elegido}

El lenguaje que he elegido para la realización de este proyecto ha sido \textit{Python}, por ser muy fácil de aprender y además tiene un gran número de bibliotecas para ciencia de datos.\\

Es cierto también que \textit{Python} es un lenguaje interpretado y bastante lento si tenemos el tiempo de ejecución, por eso la mayoría de las bibliotecas de cálculo, álgebra o ciencia de datos que realizan muchos cálculos están implementadas en otros lenguajes como \textit{C} o \textit{CUDA}. Existen alternativas muy buenas como por ejemplo \textit{Julia}, \textit{Go} o \textit{Rust} que son lenguajes compilados y con rendimiento muy parecido a \textit{C} o \textit{C++}, pero no tienen un ecosistema de bibliotecas para ciencias de datos tan grande como el de \textit{Python}. Otra alternativa interesante sería \textit{Node}, una runtime para \textit{JavaScript} (con la cual podríamos usar el mismo lenguaje para programar el \textit{front-end} y el \textit{back-end}) muy potente, pero pasa lo mismo que con lenguajes anteriores. El ecosistema de \textit{Node} está más bien enfocado a desarrollo de microservicios.

\subsection{Gestor de dependencias}

El gestor de dependencias es una herramienta muy útil hoy en día para poder manejar las dependencias de una aplicación o un proyecto de forma sencilla, usando las órdenes del mismo. Además nos permite instalar todas las bibliotecas usando únicamente una orden. Esto claramente ayuda a mantener la reproducibilidad de la aplicación ya que facilita bastante la ejecución de la misma aplicación en otra máquina (como por ejemplo en el entorno de Integración continua, en la máquina de entrenamiento de modelos o en la que desplegaremos un microservicio). En Python, los dos principales gestores de dependencias son \enquote{Pipenv} y \enquote{Poetry}.Ambas herramientas gestionan por debajo un entorno virtual aislado que contiene las dependencias instaladas.\newline

Algo que también hay que tener en cuenta es la versión del interprete que estemos usando, el usuario final debe de saber en qué versión o versiones del intérprete puede ejecutar el software sin problemas. Esto con las dos herramientas anteriores no es problema, pues en el mismo fichero de dependencias se puede especificar la versión o versiones del intérprete necesaria para poder ejecutar el software sin problemas.

\subsubsection*{Pipenv}

Pipenv se define como una herramienta que apunta a traer todo lo mejor del mundo del empaquetado al mundo Python \cite{pipenv}. Usa un fichero con sintaxis TOML para registrar las dependencias cuyo nombre es Pipfile.

\subsubsection*{Poetry}

Poetry es una herramienta cuya popularidad está creciendo un montón actualmente en la comunidad de Python. Es utilizada únicamente para manejar las dependencias de cualquier proyecto de forma muy sencilla. Además, cuenta con una documentación muy buena y clara \cite{poetry}.

\subsubsection*{Comparación de ambas herramientas}

Para comparar ambas herramientas para el manejo de dependencias, voy a utilizar el tiempo que tardan en instalar las librerías usando el fichero \textit{lock}. Las librerías que he usado como dependencias han sido \textit{PyTorch} y \textit{FastAPI} y de dependencias de desarrollo \textit{pylint} y \textit{pytest}. En la siguiente tabla se pueden ver los resultados:

\begin{table}[h]
\begin{tabular}{|c|c|c|}
\hline
                     & \textbf{Pipenv (s)} & \textbf{Poetry (s)} \\ \hline
\textbf{Ejecución 1} & 31,938          & 22,009          \\ \hline
\textbf{Ejecución 2} & 38,528          & 23,068          \\ \hline
\textbf{Ejecución 3} & 33,63           & 23,111          \\ \hline
\textbf{Ejecución 4} & 35,452          & 21,407          \\ \hline
\textbf{Ejecución 5} & 33,406          & 20,901          \\ \hline
\textbf{Media}       & 34,59           & 22,10           \\ \hline
\end{tabular}
\centering
\caption{Tiempo (medido en segundos) que tarda cada herramienta en instalar las dependencias anteriores.}
\label{tab:poetryvspipenv}
\end{table}

Como se puede ver en el Cuadro \ref{tab:poetryvspipenv}, Poetry es siempre más rápido que Pipenv, por lo tanto usaré Poetry para manejar las dependencias del proyecto. Además me gustaría añadir que en el momento del testeo, intenté instalar la librería \textit{Black}, pero Pipenv me dió problemas debido a que a veces no es capaz de resolver bien las dependencias, cosa que con Poetry no pasó. He hecho el test de tiempo debido a que necesito que se instalen las dependencias lo más rápido posible para que la Integración Continua dure lo menos posible en ejecutarse y para que se despliegue rápido la aplicación, ya que la instalación de dependencias debe realizarse en el menor tiempo posible para optimizar los pipelines de Integración Continua y de Despliegue Continuo o hacer que alguien que intente ejecutar nuestro proyecto en otra máquina, lo consiga lo antes posible.

\subsection{Gestor de tareas}

Para seguir ayudando con la reproducibilidad del proyecto, necesitamos un gestor que contenga las órdenes necesarias para testear o arrancar la aplicación o lanzar el entrenamiento de modelos. Esto nos permite que nuestro proyecto se pueda ejecutar en otra máquina de forma sencilla, ejecutando una sola orden que ya lo hace todo por nosotros. Poetry no trae por defecto ningún gestor de tareas, al contrario que Pipenv que sí lo trae. Pero por suerte, existe una herramienta que nos añade un task manager en Poetry llamada \textit{Poethepoet}. Esta herramienta nos permite lanzar una tarea tan solo escribiendo \textit{poe $<taskname>$} \cite{poethepoet}.

\subsection{Plataforma de CI/CD}

La finalidad de esta plataforma es ejecutar los tests y realizar el despliegue cada vez que se produzca un incorporación de código al repositorio de GitHub. Se usará para controlar los cambios en el código. En la CI, lo que hace es ejecutar los tests para comprobar que todo funcione bien. Si la CI tiene éxito, los cambios se aprueban y se combinan en el repositorio compartido y se despliega la aplicación en la máquina destino. Esto sería así en DevOps, pero en MLOps tenemos que entrenar los modelos antes de desplegarlos, por lo cuál el workflow se hace un poco más complejo.

\begin{figure}[h]
	\includegraphics[scale=0.4]{imagenes/03_Estado_del_arte/ci-cd-flow.png}
	\centering
	\caption{Flujo de trabajo CI/CD. \cite{cicd}}
\end{figure}

Casi todas las plataformas de CI/CD tienen un plan de pago pero también tienen un plan gratuito. Mi criterio al seleccionar una plataforma de CI/CD concreta es gastar el menor dinero posible. Por ejemplo Travis, una plataforma que antes era gratuita y es muy rápida, otorgan ahora 10.000 créditos de los cuales, debido al uso que le di en una asignatura, solo me quedan 1650, por lo que esta plataforma queda descartada. En el caso de CircleCI, otorga 2.500 créditos gratis al mes, lo cual limitaría el número de subidas de código que puedo realizar en un mes y puede ser que en algún momento se me olvide esto y me pase del plan gratuito y no pueda volver a ejecutar el pipeline de CI/CD en lo que quede de mes.\newline

Existen algunas plataformas gratuitas como por ejemplo Shippable, pero es demasiado lenta. GitHub Actions también es gratuita para repositorios públicos, como es el caso por lo que estaría interesante usarlo para comprobaciones básicas del repositorio, como por ejemplo comprobar la presencia de archivos importantes del mismo o comprobar la ortografía del README.\newline

Una plataforma a la que seguro le daré uso será Jenkins, tal vez una de las plataformas de CI/CD más populares escrita en Java. Esta plataforma es muy configurable y además dispone de un montón de plugins desarrollados por la comunidad que añaden un montón de utilidades. Para hacerlo funcionar, tenemos que instalarlo en un computador, por lo que sería necesario una instancia en alguna plataforma cloud para tenerlo disponible el mayor tiempo posible o incluso se podría ejecutar en una Raspberry Pi, lo único que necesitamos para que funcione es una Máquina Virtual Java. En mi caso tengo 150\$ en la plataforma AWSEducate, por lo que podría desplegar una instancia con Jenkins estando disponible el mayor tiempo posible sin problemas.\newline

Todas las plataformas anteriores disponen de un fichero de configuración que se almacena en el repositorio en el que hay que describir la secuencia de órdenes para poder ejecutar los tests o hacer el despliegue, además de especificar el contenedor en el que ese ejecuta, versiones del lenguaje, etc.

\subsection{Framework para MLOps}

Lo que necesito de un framework para MLOps es que registre los experimentos realizados (hiperparámetros usados y resultados obtenidos) y además nos facilite la reproducibilidad de los mismos, es decir, que se obtengan los mismos resultados independientemente de la máquina/entorno en el que se ejecuten.\newline

Existe un framework llamado SnapperML que facilita todo lo anterior. Permite almacenar los hiperparámetros de los modelos en formato YAML, además de tener un control de la estocasticidad (recoge información sobre la semilla utilizada para los diferentes generadores de números aleatorios). También permite el \textit{tracking} de experimentos usando \textit{MLFlow}, una herramienta que permite registrar métricas, hiperparámetros y artefactos y nos permite acceder a ellos a través de una interfaz web bastante intuitiva. Además de todo lo anterior, usa una librería llamada \textit{Optuna} para la optimización de hiperparámetros y \textit{Ray} para la ejecución en cluster \cite{snapperml}.\newline

Como se puede ver, este framework es todo lo que necesito para conseguir la reproducibilidad de los modelos de Machine Learning, pues la reproducibilidad del software ya está conseguida gracias a los gestores de dependencias y de tareas.

\subsection{Aprovisionamiento de Infraestructura Virtual}

Otro detalle que no se nos puede olvidar, es la reproducibilidad de la infraestructura para que la misma pueda ser creada y aprovisionada en otras cuentas de la misma plataforma cloud de forma rápida y fácil. ¿Para qué queremos aprovisionar una infraestructura? En este caso, es necesaria por ejemplo para aprovisionar una instancia de EC2 con \textit{Jenkins}.\newline

Para esto existen dos alternativas muy populares a día de hoy: \textit{Pulumi} \cite{pulumi} y \textit{Terraform} \cite{Terraform}. Ambas nos dejan describir la Infraestructura como Código (IaC), lo que nos facilita un montón la reproducibilidad. Las diferencias entre ambos son las siguientes:

\begin{itemize}
	\item \textbf{Lenguaje de programación}: \textit{Terraform} usa Hashicorp Configuration Language (HCL), un lenguaje propio declarativo para poder describir nuestra infraestructura. Con \textit{Pulumi}, se puede usar un lenguaje de programación de propósito general (\textit{TypeScript}, \textit{JavaScript}, \textit{Python}, \textit{Go} o \textit{C\#}). Usar un lenguaje de propósito general puede ser más cómodo que usar uno declarativo, pues por ejemplo nos permite usar estructuras condicionales si llega a ser necesario.
	\item \textbf{Documentación}: a día de hoy \textit{Terraform} tiene una documentación más completa que \textit{Pulumi}. Es por eso que a lo mejor también tiene una mayor comunidad que \textit{Pulumi}.
	\item \textbf{Modularidad}: \textit{Terraform} presenta una mejor modularidad que \textit{Pulumi}, esto es gracias a que presenta componentes que son perfectamente reusables.
\end{itemize}

En conclusión, ambas herramientas están muy bien diseñadas, cada una tiene sus pros y sus contras. No obstante, he decidido usar \textit{Terraform} para aprovisionar las máquinas necesarias para este proyecto debido a que tener una buena documentación es algo que debería ser esencial hoy en día para poder aprender a usar dicha herramienta sin problemas o consultarla cuando algo falle.

\subsection{Contenedores}

Además de lo dicho en el capítulo de introducción, los contenedores no se ejecutan sobre un hipervisor que funcionan sobre el Sistema Operativo Host, como sí hacen las máquinas virtuales, sino que se ejecutan como un proceso más y comparten el Kernel. Esto hace que las imágenes de los contenedores sean ligeras y además no degrade el rendimiento de la aplicación al no necesitar una capa de abstracción sobre el Sistema Operativo Host.\newline

Otra gran característica de esta tecnología es que se puedan desarrollar nuevas actualizaciones del propio contenedor de forma fácil, pues utilizan por debajo un sistema de versionamiento.

\begin{figure}[H]
	\includegraphics[scale=0.3]{imagenes/03_Estado_del_arte/containers.png}
	\centering
	\caption{Diferencia entre un contenedor y una Máquina Virtual.}
\end{figure}

Entre los \textit{container engines} más populares tenemos \textit{Docker} \cite{docker} y \textit{Podman} \cite{podman}. Recientemente, \textit{Docker} ha sido deprecado en \textit{Kubernetes} (una tecnología para orquestar contenedores ejecutándose en un cluster) debido a problemas de seguridad que presenta su diseño (usa un demonio para funcionar y ejecuta los contenedores con permisos root). \textit{Podman} arregla los problemas que presenta \textit{Docker}, pues no necesita de un demonio para ejecutarse y es \textit{rootless} por tanto es un \textit{engine} más seguro.

\section{Modelos de Machine Learning y Deep Learning}

En esta sección se describirán los modelos de Machine Learning o Deep Learning que se planean usar en los experimentos sobre el dataset. Debemos de tener en cuenta que nos estamos enfrentando a un problema de clasificación, es decir, tenemos que entrenar modelos que intenten predecir un valor discreto \{0, 1\}. El 0 indica que la señal no es de un muón y el 1 indica que sí hay muón registrado en esa señal. Las variables que nos ayudan a predecir son las medidas de la señal en distintos momentos, es una serie temporal. En este proyecto probaré algunos modelos estado del arte en el campo de análisis de series temporales.

\subsection{Concepto básicos}

Algunos de los conceptos básicos sobre Machine Learning o Deep Learning son los siguientes:

\begin{itemize}
	\item \textbf{Dataset}: conjunto de datos que disponemos para entrenar el modelo predictivo. Este dataset se divide en 3 subconjuntos disjuntos, para entrenenar, validar y evaluar el modelo.
	\item \textbf{Training Dataset}: contiene los datos que usaremos para ajustar el modelo.
	\item \textbf{Validation Dataset}: los datos contenidos en este dataset se utilizan para la selección de hiperparámetros más convenientes para entrenar el modelo.
	\item \textbf{Test Dataset}: se utiliza para evaluar la potencia predictiva del modelo.
	\item \textbf{Alto sesgo (high bias)}: esto se produce cuando el poder predictivo del modelo en el conjunto de entrenamiento es malo y por tanto tampoco es capaz de hacer buenas predicciones fuera del conjunto de entrenamiento.
	\item \textbf{Alta variabilidad (high variability)}: esto se produce cuando el modelo se comporta muy bien en el conjunto de entrenamiento, pero funciona muy mal en el conjunto de test (hace predicciones muy malas). Esto es consecuencia del \textit{overfitting}, que se produce porque el modelo se sobreajusta al conjunto de entrenamiento y por tanto no rinde bien en el conjunto de test. Esto suele pasar porque no tenemos datos suficientes para hacer una predicción en condiciones o porque hemos usado un modelo muy complejo.
\end{itemize}

Siempre el conjunto de entrenamiento debe ser mayor que los demás, ya que usar un número muy grande de datos hace que el modelo generado del entrenamiento sea mejor. En los capítulos posteriores explicaré los modelos estado del arte que usaré y los mejores algoritmos de entrenamiento de modelos actuales. También hablaré sobre como podemos evitar el overfitting usando un método muy sencillo llamado regularización.

\subsection{Redes neuronales}

Las redes neuronales son un modelo muy usado para aproximar funciones y en muchos problemas más. Fue inventada en los años 80, pero no era muy usada debido a que no había computadores tan potentes como los de ahora. Gracias a las capacidades de computación modernas, se han podido hacer Redes Neuronales más profundas que ofrecían mejores resultados.\newline

Este modelo de aprendizaje automático se puede usar tanto para regresión lineal (cuando el valor que queremos predecir es continuo) como para regresión logística o clasificación (intentamos predecir un valor discreto que representa la clase a la que pertenece la instancia del dataset). Vamos a pasar a ver cómo funciona este modelo tan popular, empezando por un modelo muy básico.

\begin{figure}[H]
	\includegraphics[scale=0.25]{imagenes/03_Estado_del_arte/simplenn.png}
	\centering
	\caption{Red neuronal básica \cite{ng}.}
	\label{fig:simlestnn}
\end{figure}

En la Figura \ref{fig:simlestnn} se puede observar una Red Neuronal muy simple, de hecho este caso específico de red es lo que se llama regresión lineal o logística (dependiendo del rango que tome el valor que intentamos predecir). Una neurona no es más que una función cuya salida es $h_{W,b}(x)=f(W^Tx+b)$, el parámetro $b$ corresponde a las unidades de \textit{bias} y debe ser aprendido al igual que $W$ que representa los pesos. La función $f$ es la función de activación, las más comunes son sigmoide (\ref{eq:sigmoid}), tangente hiperbólica (\ref{eq:tanh}) y RELU (\ref{eq:RELU}). El vector $x$ es el vector de datos, es decir, contiene el dataset. Expresado de otra forma más sencilla, la salida es la siguiente:\newline

$$h_{W,b}=f(W_1 x_1+ W_2 x_2 + W_3 x_3 + b)$$

Las funciones de activación de las que hablé pueden ser las siguientes:

\begin{equation}
	f(x) = \frac{1}{1+e^{-x}}
	\label{eq:sigmoid}
\end{equation}

\begin{equation}
	f(x) = \frac{e^x - e^{-x}}{e^x + e^{-x}}
	\label{eq:tanh}
\end{equation}

\begin{equation}
	f(x) = \max (0, x)
	\label{eq:RELU}
\end{equation}

Con el modelo anterior podríamos predecir un valor o la probabilidad de que una instancia del Dataset pertenezca a una clase u otra (usando por ejemplo la sigmoide que devuelve un valor entre 0 y 1). Ahora que sabemos lo básico, podemos pasar a explicar otra un poco más compleja para comprender como funciona una red neuronal. \newline

\begin{figure}[H]
	\includegraphics[scale=0.2]{imagenes/03_Estado_del_arte/complexnn.png}
	\centering
	\caption{Red neuronal un poco más compleja \cite{ng}.}
	\label{fig:complexnn}
\end{figure}

En la Figura \ref{fig:complexnn} podemos ver una Red Neuronal más compleja que la anterior la cual nos permite aproximar funciones también más complejas. Tanto en esta como en la anterior podemos observar que el tamaño de la entrada en la red es el tamaño de una instancia del dataset + 1 (la unidad de \textit{bias}). El tamaño del output, en un problema de clasificación depende del número de clases que tengamos, en este caso sirve para 2. En este caso, para realizar una predicción deberíamos usar el algoritmo de propagación hacia delante:

\begin{align*}
&a^{(1)} = x^{(i)} \\
&z^{(2)} = W^{(1)}a^{(1)} + b^{(1)} \\
&a^{(2)} = f(z^{(2)}) \\
&z^{(3)} = W^{(2)}a^{(2)} + b^{(2)} \\
&a^{(3)} = f(z^{(3)}) \\
&z^{(4)} = W^{(3)}a^{(3)} + b^{(3)} \\
&h_{W,b}(x) = a^{(4)} = f(z^{(4)})\\
\end{align*}

Una vez obtenido el resultado del algoritmo de propagación hacia delante, podemos obtener la probabilidad de que la instancia pertenezca a cada clase usando \textbf{softmax}:

\begin{align*}
\sigma (z)_j = \frac{e^{z_j}}{\sum^{K}_{k=1} e^{z_k}}
\end{align*}

\subsubsection{Descenso del gradiente}

Anteriormente se ha explicado cómo funciona una red neuronal para predecir, pero ¿Cómo se entrenan para tal objetivo? Primero tenemos que especificar la función que queremos optimizar o dicho de otras palabras, la función que calcula el error en las predicciones y que queremos optimizar. Puede ser el Mean Squared Error (MSE):

\begin{align*}
	&J(W,b;x^{(i)},y^{(i)}) = \frac{1}{m} \sum^m_{i=1} (h_{W,b}(x^{(i)}) - y^{(i)} )^2
\end{align*}

En este caso, el error se calcula como la diferencia entre el valor predecido y el valor correcto. Otra función que podemos usar es la de Cross Entropy (CE):

\begin{align*}
	&J(W,b;x^{(i)},y^{(i)}) = - \frac{1}{m}\left(\sum^m_{i=1} y^{(i)} \cdot \log{h_{W,b}(x^{(i)}})\right)
\end{align*}

El caso específica de CE para dos clases se calcula así:

\begin{align*}
	&J(W,b;x^{(i)},y^{(i)}) = - \frac{1}{m} \left( \sum^m_{i=1} y^{(i)} \cdot \log (h_{W,b}(x^{(i)})) + (1 - y^{(i)}) \cdot \log (1 - h_{W,b}(x^{(i)})) \right)
\end{align*}

Una vez elegida la función de pérdida, debemos optimizarla usando el algoritmo llamado \textit{descenso del gradiente}. Dicho algoritmo actualiza los pesos en cada iteración de la siguiente forma en los modelos básicos de regresión (la red neuronal más básica):

\begin{align*}
	&W_j = W_j - \alpha \frac{\partial}{\partial W_j} J(W,b)\\
	&b = b - \alpha \frac{\partial}{\partial b} J(W,b)\\
\end{align*}

El parámetro $\alpha$ representa el ratio de aprendizaje y es otro parámetro que debemos elegir usando el conjunto de validación. Si es muy alto, el aprendizaje puede fallar y si es muy bajo el entrenamiento tarda un montón en finalizar. Sin embargo, calcular las derivadas parciales en una red neuronal con más capas es un poco más complejo:

\begin{align*}
	&W^{(l)}_{ij} = W^{(l)}_{ij} - \alpha \frac{\partial}{\partial W^{(l)}_{ij}} J(W,b)\\
	&b^{(l)}_i = b^{(l)}_i - \alpha \frac{\partial}{\partial b^{(l)}_i} J(W,b)\\
\end{align*}

Para calcular las derivadas parciales de forma eficiente, se usa el algoritmo de \textit{propagación hacia detrás}:\\

\begin{algorithm}[H]
	\SetAlgoLined
	
	Calcular $a^{(l)}$ $\forall l \in \{2, ..., L\}$ usando el algoritmo de \textit{propagación hacia delante}
	
	Para la capa de salida, calculamos: $\delta^{(L)} = \frac{\partial}{\partial z^{(L)}} \frac{1}{2} || y - h_{W,b}(x) ||^2 = (a^{(L)}- y) \odot f'(z^{(L)})$
	
	\For{$l \in \{L-1, ..., 2\}$}{
		\nonl $\delta^{(l)} = ((W^{(l)})^T \delta^{(l+1)}) \odot f'(z^{(l)})$
	}
	
	Calcular las derivadas parciales:\\
	\nonl $\frac{\partial}{\partial W^{(l)}} J(W,b) = \delta^{(l+1)}(a^{(l)})^T$\\
	\nonl $\frac{\partial}{\partial b^{(l)}} J(W,b) = \delta^{(l+1)}$
	
	\caption{Backpropagation Algorithm}
\end{algorithm}

\subsection{Algoritmos de aprendizaje}

\subsubsection{Descenso del gradiente estocástico}

Con un dataset muy grande, como es el caso, el descenso del gradiente normal tarda mucho en converger a una solución óptima. En ese caso, puede resultar beneficioso el uso del Stochastic Gradient Descent (SGD).\newline

Este algoritmo consiste en tomar un subconjunto (mini-batch) de datos aleatorio y mucho más pequeño que el propio dataset en cada iteración. Después de haber selecciondo aleatoriamente dicho subconjunto, realizamos una iteración de Gradient Descent (GD) sobre el mini-batch.

\subsubsection{Adaptative Moment Estimation (Adam)}

Adam hace uso de \textit{momentum}, un método de actualiza los pesos no solo teniendo en cuenta el gradiente actual, sino también todos los anteriores. Esto hace que el ajuste sea más suave.

\begin{align}
	&V_{dW} \leftarrow \beta \cdot V_{dW} + (1 - \beta) \frac{\partial}{\partial W} J(W,b)\\
	&V_{db} \leftarrow \beta \cdot V_{db} + (1 - \beta) \frac{\partial}{\partial b} J(W,b)\\
	&W \leftarrow W - \alpha V_{dW}
	&b \leftarrow b - \alpha V_{db}
\end{align}

Y también usa \textit{RMSprop}, un método parecido a \textit{momento} pero la diferencia es que usa el segundo momento del gradiente para hacer el ajuste más suave.

\begin{align}
	&S_{dW} \leftarrow \beta_2 \cdot S_{dW} + (1 - \beta_2) \left( \frac{\partial}{\partial W} J(W,b) \right)^2\\
	&S_{dW} \leftarrow \beta_2 \cdot S_{dW} + (1 - \beta_2) \left( \frac{\partial}{\partial b} J(W,b) \right)^2\\
	&W \leftarrow W - \alpha \frac{\frac{\partial}{\partial W} J(W,b)}{\sqrt{S_{dW} + \epsilon}}
	&b \leftarrow b - \alpha \frac{\frac{\partial}{\partial b} J(W,b)}{\sqrt{S_{db} + \epsilon}}
\end{align}

Adam hace uso de los métodos anteriores para conseguir un ajuste más suave que el SGD, llegando antes a una solución óptima y mejor en muchos casos.

\begin{align}
	&t \leftarrow iteration\_number\\
	&V_{dW} \leftarrow \beta_1 \cdot V_{dW} + (1 - \beta_1) \frac{\partial}{\partial W} J(W,b)\\
	&V_{db} \leftarrow \beta_1 \cdot V_{db} + (1 - \beta_1) \frac{\partial}{\partial b} J(W,b)\\
	&S_{dW} \leftarrow \beta_2 \cdot S_{dW} + (1 - \beta_2) \left( \frac{\partial}{\partial W} J(W,b) \right)^2\\
	&S_{dW} \leftarrow \beta_2 \cdot S_{dW} + (1 - \beta_2) \left( \frac{\partial}{\partial b} J(W,b) \right)^2\\
	&V_{dW}^{corrected} \leftarrow \frac{V_{dW}}{(1 - \beta_1^t)}
	&V_{db}^{corrected} \leftarrow \frac{V_{db}}{(1 - \beta_1^t)}\\
	&S_{dW}^{corrected} \leftarrow \frac{S_{dW}}{(1 - \beta_2^t)}
	&S_{db}^{corrected} \leftarrow \frac{S_{db}}{(1 - \beta_2^t)}\\
	&W \leftarrow W - \alpha \frac{V_{dW}^{corrected}}{\sqrt{S_{dW}^{corrected}} + \epsilon}
	& b \leftarrow b - \alpha \frac{V_{db}^{corrected}}{\sqrt{S_{db}^{corrected}} + \epsilon}
\end{align}

En este algoritmo tenemos cuatro hiperparámetros, $\beta_1$, $\beta_2$, $\alpha$ y $\epsilon$ \cite{kingma2017adam}. Los valores típicos de estos hiperparámetros son los siguientes:

\begin{align*}
	\beta_1 &= 0.9\\
	\beta_2 &= 0.999\\
	\epsilon &= 10^{-8}
\end{align*}

\subsection{Mejoras para el modelo de Red neuronal}

\subsubsection{Regularización}

La regularización sirve para evitar el riesgo de un sobreajuste de un modelo complejo a los datos de entrenamiento. El mecanismo es muy sencillo, simplemente se añade una penalización a la función de pérdida para que el modelo resultante sea menos propenso a \textit{overfitting}:

\begin{align*}
	&J(W,b;x,y) = J(W,b;x,y) + \lambda \sum_{j=1}^n W_j^2\\
\end{align*}

En este caso, $\lambda$ es un hiperparámetro que reprenta cuánto queremos penalizar la flexibilidad del modelo, por lo que también se elige usando el conjunto de validación. Si tomamos un valor muy alto para $\lambda$, los parámetros $W$ y $b$ serán muy próximos a 0 y tal vez genere un problema de alto sesgo. Por el contrario, un valor muy cercado a 0 apenas penalizaría el modelo y la regularización no tendría efecto. Si usamos el valor correcto de $\lambda$, pagaríamos con un poco de \textit{bias} para una mejoría mayor en la \textit{variability}. En cada paso del gradiente descendiente se realiza ahora lo siguiente:

\begin{align*}
	&\frac{\partial}{\partial W^{(l)}}_{ij} J(W,b) = \left[ \frac{\partial}{\partial W^{(l)_{ij}}} J(W,b;x^{(i)},y^{(i)}) \right] + 2 \lambda W^{(l)}_{ij}\\
	&\frac{\partial}{\partial b^{(l)}}_i J(W,b) = \frac{\partial}{\partial b^{(l)}_i} J(W,b;x^{(i)},y^{(i)})\\
\end{align*}

\subsubsection{Batch Normalization}

Gracias a este método podemos acelerar el aprendizaje del modelo y además generar redes neuronales más profundas y con mejores predicciones. Este método lo que hace normalizar las activaciones para acelerar el aprendizaje. Básicamente coge las varianza y la media para hacer que $\mu = 0$ y $\sigma = 1$:

\begin{align}
	&\mu \leftarrow \frac{1}{m} \sum_{i=1}^m z_i\\
	&\sigma^2 \leftarrow \frac{1}{m} \sum_{i=1}^m (z_i - \mu)^2\\
	&\hat{z_i} \leftarrow \frac{z_i - \mu}{\sqrt{\sigma ^ 2 + \epsilon}}\\
	&\hat{z_i} \leftarrow \gamma \hat{z_i} + \beta 
\end{align}

Los parámetros $\gamma$ y $\beta$ deben ser aprendidos. $\epsilon$ sirve para evitar la inestabilidad numérica debido a una división por 0, por tanto su valor debe ser muy pequeño \cite{DBLP:journals/corr/IoffeS15}.

\subsection{Autoencoders}

Los autoencoders son redes neuronales utilizada, entre otras cosas, para reducir la dimensionalidad de los datos. A veces da mejores aprender de estos datos de dimensión reducida, es por eso que sería interesante probar este modelo. Estas redes neuronales se entrenan para reconstruir su \textit{input}, es por eso que están formadas por dos partes: un \textit{encoder} y un \textit{decoder}. El \textit{encoder} se encarga de reducir los datos de la entrada a una dimensión menor (al igual que el algoritmo PCA). Los datos comprimidos en menor dimensión reciben el nombre de código. La parte del \textit{decoder} se encarga de reconstruir los datos originales con la menor pérdida posible. Estas dos redes neuronales que forman el autoencoder se suelen entrenar juntas.

\begin{figure}[H]
	\includegraphics[scale=0.3]{imagenes/03_Estado_del_arte/autoencoder.png}
	\centering
	\caption{Ejemplo de autoencoder \cite{vae}.}
	\label{fig:autoencoder}
\end{figure}

Este problema, se suele definir formalmente como aprender las funciones $A: \mathbb{R}^n \rightarrow \mathbb{R}^p$ (encoder) y $B: \mathbb{R}^p \rightarrow \mathbb{R}^n$ (decoder) que satisfaga

\begin{align}
	&\argmin_{A,B}E[\Delta (x, B \circ A(x))]
\end{align}

Donde E es la media y $\Delta$ es la función que calcula el error de reconstrucción, que mide la distancia entre la salida del docoder y el input.\newline

Este tipo de modelos se suele usar mucho para clasificación o predicción de series temporales, para reducción de dimensionalidad, para crear Generative Adversarial Networks, etc. Existen varios tipos de autoencoders que serán explicados a continuación.

\subsubsection{Sparse Autoencoders}

Para evitar que tras el entrenamiento se obtengan A y B como funciones identidades, podemos forzar a que las activaciones de las capas ocultas sean dispersas. Para ello, podríamos usar regularización $L_1$, la que hemos visto antes. La diferencia es que la aplicaríamos sobre las activaciones en lugar de los pesos. Por tanto el objetivo sería

\begin{align}
	&\argmin_{A,B}E[\Delta (x, B \circ A(x))] + \lambda \sum_i |a_i|
\end{align}

Donde $a_i$ es el valor de la activación de la i-ésima capa oculta. Otra forma de hacerlo sería usando la divergencia de Kullback-Leibler (mide la distancia entre dos funciones de distribución) asumiendo que las activaciones siguen una distribución de Bernuilli con probabilidad $p$. La función de pérdida pasaría a ser

\begin{align}
	&\argmin_{A,B}E[\Delta (x, B \circ A(x))] + \sum_j KL(p||\hat{p}_j)\\
	&\hat{p}_j = \frac{1}{m} \sum_i a_j^{(2)}(x^{(i)})
\end{align}

Donde $a_i(x)$ representa el valor de la activación dada una entrada $x$. En este caso, el término de regularización es $p$, el cuál es forzado a aproximarse a $\hat{p}$. Normalmente $p$ suele ser un valor cercano a 0 \cite{ng, bank2020autoencoders}.

\subsubsection{Stacked Autoencoders}

Los autoencoders dispersos solo sirven para redes neuronales de tres capas, una para el encoder, otra para el código y otra para el decoder. Pero, ¿qué pasa si los datos que queremos aprender patrones más complejos en lugar de patrones lineales? La solución es simple, basta con añadir más capas al autoencoder en la parte del encoder, del decoder o en ambas.\newline

Las redes neuronales son un modelo muy potente y propensas a \textit{overfitting}, por eso la regularización es siempre esencial en estos casos.

\subsubsection{Varitional Autoencoders}

Este tipo de autoencoders trata de describir la generación de datos a través de una distribución de probabilidad. Por tanto, en lugar de mapear la entrada en un vector de tamaño fijo, la vamos a mapear a una distribución. Por consecuencia, se usarán dos vectores uno que representa la media y el otro la desviación típica. La función de pérdida para este tipo de Autoencoder es la siguiente:

\begin{align}
	&\mathcal{L}(\theta, \phi; x^{(i)}) = \mathbb{E}_{q_\phi(z|x^{(i)})} \left[ \log p_\theta(x^{(i)}|z) \right] - KL(q_\phi(z|x_i)||p_\theta(z))
\end{align}

En la anterior función, $\mathbb{E}_{q_\phi(z|x^{(i)})} \left[ \log p_\theta(x^{(i)}|z) \right]$ representa la función de pérdida (como por ejemplo el MSE que usamos antes en la sección sobre redes neuronales).\newline

Para ejecutar la propagación hacia detrás, $\mu$ (vector que representa la media de la distribución de los datos) y $\sigma$ (vector que representa la desviación típica de la distribución de los datos) se pasan a un único vector latente $z$ usando el truco de reparametrización. Los más usual es aproximar $p(z|x)$ a una distribución Gaussiana $q_\phi(z|x) = \mathcal{N}(g(x),h(x))$ donde $g(x)$ es la media y $h(x)$ es la covarianza de la distribución. Por tanto, podemos aplicar el truco de reparametrización esta manera:

\begin{align}
	&z = h(x) \xi + g(x)
\end{align}

Donde $\xi \sim \mathcal{N}(0,\mathbf{I})$ sigue una distribución normal. La KL divergencia en este caso se calcula de la siguiente manera \cite{kingma2014autoencoding}:

\begin{align}
	&KL(q_\phi(z|x_i)||p_\theta(z)) = \frac{1}{2} \sum^J_{j=1} \left( 1 + \log ((\sigma_j)^2) - (\mu_j)^2 - (\sigma_j)^2 \right)
\end{align}

\begin{figure}[H]
	\includegraphics[scale=0.3]{imagenes/03_Estado_del_arte/vae-gaussian.png}
	\centering
	\caption{El proceso del truco de reparametrización explicado gráficamente \cite{weng2018VAE}.}
	\label{fig:gaussianvae}
\end{figure}

\subsubsection{Autoencoder para clasificación}

Dado el error de de reconstrucción como $R(x,\hat{x})$ y el error de clasificación $\mathcal{L}(y,\hat{y})$, podemos usar la función de pérdida $\tilde{\mathcal{L}} = \mathcal{L}(y,\hat{y}) + \lambda R(x,\hat{x})$ para entrenar un autoencoder que clasifica a partir de las variables generadas en el espacio latente. En esta nueva función, el error de reconstrucción pasa a ser el valor de regularización.

\begin{figure}[H]
	\includegraphics[scale=0.3]{imagenes/03_Estado_del_arte/classifierautoencoder.png}
	\centering
	\caption{Representación gráfica de un Autoencoder clasificador \cite{bank2020autoencoders}.}
	\label{fig:classiferautoencoder}
\end{figure}

\subsection{Redes Neuronales Convolucionales}

Las redes neuronales convolucionales son un modelo muy ampliamente usado hoy en día para detección de patrones en imágenes (datos 2D) el cual es muy potente y da buenos resultados. En este trabajo no estamos clasificando imágenes pero estamos buscando patrones en un vector 1D con medidas de señales para detectar si era señal muónica o no. Este modelo se puede aplicar también sobre 3D.\\

La idea básica de una capa convolucional consiste en un vector que llamaremos \textit{Kernel} que contiene los pesos e iremos moviendo a través del vector.\\

\begin{figure}[H]
	\includegraphics[scale=0.3]{imagenes/03_Estado_del_arte/conv1d.png}
	\centering
	\caption{Explicación gráfica de las Convoluciones Unidimensionales.}
	\label{fig:conv1d}
\end{figure}

Cada red de la arquitectura convolucional hace que el vector vaya perdiendo tamaño a la vez que vamos detectando patrones en el mismo. Por ejemplo, un vector de tamaño 6 usando un \textit{Kernel} de tamaño 3 quedaría en un vector de tamaño 4. Aunque con \textit{padding} esto se puede evitar (viene explicado más adelante).\\

Con este tipo de capas, también podríamos diseñar un autoencoder y estaría muy interesante ver si es capaz de competir con el Autoencoder diseñado con una red neuronal totalmente conectada. De hecho, los autoencoders convolucionales suelen usarse mucho y tienen un gran potencial en segmentación de imágenes y como GANs. Un ejemplo de uso es en U-NET, una CNN diseñada para segmentación de imágenes médicas \cite{DBLP:journals/corr/RonnebergerFB15}.\\

Este tipo de modelo también puede combinarse con redes neuronales totalmente conectadas para realizar la clasificación, por lo que también estaría interesante comparar el potencial entre simplemente usar únicamente capas totalmente conectadas o usar una combinación.\\

Una capa de convolución puede tener más de un \textit{Kernel}, así que podemos tener de salida dos vectores o más lo cual puede llegar a detectar más patrones de los datos de entrada.

\subsubsection{Modificaciones en las convoluciones}

Una de las modificaciones más importantes es el \textit{padding}. Esta modificación permite que al realizar una convolución no se reduzca el tamaño del vector, permitiéndonos entrenar redes convolucionales más profundas. Esto se consigue añadiendo 0s a los extremos del vector.\\

Otra modificación interesante es el \textit{stride}, la cual solo consiste en saltarse entre convolución y convolución $n$ posiciones a la derecha.

\subsubsection{Pooling layers}

Algunas arquitecturas de redes convolucionales, además de usar capas convolucionales y totalmente conectadas, también usan \textit{pooling layers}. Este tipo de capas se utilizan para acelerar la computación, pues no tienen parámetros que aprender. Los dos tipos principales de \textit{pooling} son:

\begin{itemize}
	\item \textbf{Max pooling}: consiste en un \textit{kernel} que va a ir tomando el máximo valor de la región del vector (o dato que estemos procesando) en la que se encuentra.
	\item \textbf{Average pooling}: en lugar de tomar el máximo valor, tomaremos el valor medio.
\end{itemize}

\subsubsection{ResNets}

Las \textit{Residual Networks} son una arquitectura de Red Convolucional muy potente que nos permite entrenar redes muy profundas. Un \textit{ResNet block} consta de dos o tres capas convolucionales seguidas de una suma de las activaciones de la capa anterior, tal y como se muestra en la Figura \ref{fig:resnet}. Concatenando varios \textit{Resnet Blocks} podemos crear una \textit{ResNet}.

\begin{figure}[H]
	\includegraphics[scale=0.35]{imagenes/03_Estado_del_arte/resnet-block.png}
	\centering
	\caption{Ejemplos de \textit{Resnet Blocks} \cite{resnetblocks}.}
	\label{fig:resnet}
\end{figure}


%\chapter{Planificación y presupuesto}

\section{Planificación}

Para realizar este proyecto, lo he dividido en varias etapas. La primera etapa consta de definición del problema, búsqueda de objetivos e investigación del estado del arte (el cual ha sido abordado en los Capítulos 1, 2 y 3 de este documento). La segunda etapa consistirá en un aprovisionamiento de la infraestructura necesaria para los tests del software o de los modelos predictivos. La tercera etapa consistirá en implementación y experimentación de los modelos predictivos usando uno o varios frameworks de Machine Learning o Deep Learning. Y por último, una cuarta etapa de descripción de la implementación, del despliegue y monitorización de un servicio que haga uso de un modelo predictivo y la realización de los tests de infraestructura necesarios. Esta última etapa se realizará en un trabajo futuro usando las herramientas de orquestación necesarias.\\

La metodología que usaré para llevar a cabo el proyecto será \textit{Kanban} \cite{kanban}. Esta metodología usa las llamadas \textit{Historias de Usuario} para reflejar los deseos de los clientes (o los míos), las cuales tengo en \textit{GitHub} enunciadas en los \textit{issues} de este proyecto \cite{issues}. Además de lo anterior, en esta metodología se utiliza un tablero que nos permite visualizar el flujo del trabajo. En este tablero se encuentran las tareas necesarias en tres columnas distintas: \textit{To do} (aún no se ha empezado), \textit{In progress} (se está realizando ahora mismo) y \textit{Done} (finalizada).\\

Además de los anterior, \textit{GitHub} también nos permite agrupar las \textit{issues} en \textit{milestones}, cada milestone representará una etapa de este proyecto y haré uno especial para las Historias de Usuario y otro para documentación. Y si lo de antes es poco, también nos permite añadir un tablero \textit{kanban} al repositorio. En la Figura \ref{fig:kanban} se puede observar el que yo he realizado.\\

Podemos ver que \textit{GitHub} es una gran herramienta para el desarrollo de software que nos permite trabajar cómodamente de forma individual o en grupos.

\begin{figure}[H]
	\includegraphics[scale=0.3]{imagenes/04_Planificacion/kanban.png}
	\centering
	\caption{Tablero \textit{kanban} en \textit{GitHub}. \url{https://github.com/ManuelJNunez/TFG/projects/1}}
	\label{fig:kanban}
\end{figure}

\begin{figure}[H]
  \centering
  \includegraphics[width=1.\linewidth]{imagenes/04_Planificacion/gantt.png} 
  \caption{Diagrama de Gantt del proyecto.}
  \label{fig:gantt}
\end{figure}

\section{Presupuesto}

En esta Sección se va a proceder a calcular el coste del desarrollo de este proyecto, teniendo en cuenta el coste humano y material. El sueldo medio para un ingeniero de software es de aproximadamente 8,44\euro/hora, mientras que el sueldo medio de una persona que posee un doctorado es de aproximadamente 10\euro/hora\footnote{Estimación basada en la información del OTRI \cite{otri} en base a lo establecido en el BOJA número número 11 del 13 de enero de 2021 (página 281)}. En total, he dedicado 319h a este trabajo tal y como puede verse en la Figura \ref{fig:hours}. Como se puede ver en esa Figura, en el 2020 apenas le dediqué tiempo porque tenía 5 asignaturas. Sin embargo, en 2021 he podido dedicar más tiempo a este trabajo aunque tuviera prácticas de empresa y una sola asignatura.\\

\begin{figure}[H]
  \centering
  \includegraphics[width=1.\linewidth]{imagenes/04_Planificacion/hours.png} 
  \caption{Gráfico de barras y total de horas realizadas.}
  \label{fig:hours}
\end{figure}

Sobre el coste de las instancias de \textit{EC2}, no he puesto las horas de uso porque no se muestran en ninguna parte de \textit{AWS}, por tanto he puesto el coste total teniendo en cuenta los créditos con los que empecé y los que tengo ahora mismo. El coste original estaba en dólares estadounidenses, pero lo pasé a euros\footnote{Hoy día 5 de julio de 2021 el dólar estadounidense tiene un valor de 0.84\euro.}.\\

Además de todo lo anterior, he usado el nodo llamado \textit{compute-0-1} del clúster \textit{HPMoon}, el cual cuenta con el siguiente hardware y especificaciones:

\begin{itemize}
    \item Intel\textsuperscript{\textregistered} Xeon\textsuperscript{\textregistered} CPU E5-2620 v4 @ 2.10GHz (8 núcleos por procesador; 16 hebras de procesamiento en total). Su precio actual es de 449,90\euro.
    \item 32GB memoria RAM. Supongamos un precio de 200\euro\footnote{No conozco la latencia de la memoria ni la frecuencia por tanto no puedo averiguar el precio justo (supongo que usa dos módulos de 16GiB con latencia CL18 y frecuencia de 3600MHz y tipo DDR4).}.
    \item NVIDIA GeForce\textsuperscript{\textregistered} Titan Xp GPU. Descatalogada, pero su precio era de 1.349\euro\footnote{Debido a la actual falta de \textit{stock} producido por el \textit{boom} de la minería de criptomonedas, su valor no habrá variado mucho.}. Fue lanzada en el 2017.
    \item NVIDIA Tesla K40M GPU. Fue lanzada en el 2013, por tanto también está descatalogada. Algunas unidades de esta GPU que quedan en \textit{stock} tienen el valor de 796,56\euro.
    \item Una placa base con socket LGA2011-3 (compatible con el procesador anterior) y con 2 ranuras PCI Express 3.0 podría tener un valor de 160,69\euro.
    \item Además, el clúster cuenta con 15TB de capacidad de disco duro, su valor aproximado sería de 573\euro.
    \item Suponemos un chasis de servidor de 150\euro.
\end{itemize}

En total, el coste aproximado del nodo usado para entrenar modelos podría ascender actualmente a los 3679,15\euro.\\

\setlength{\tabcolsep}{10pt} % Default value: 6pt
\renewcommand{\arraystretch}{1.5} % Default value: 1

\begin{table}[H]
\centering
\begin{tabular}{ccccc}
\hline
\textbf{Tipo Coste}       & \textbf{Concepto}                                               & \textbf{Tiempo (h)} & \textbf{Coste (\euro)} & \textbf{Total (\euro)} \\ \hline
\multirow{3}{*}{Material} & \begin{tabular}[c]{@{}c@{}}Jenkins\\ (t2.medium)\end{tabular}   & -                   & 49,90              & 49,90              \\ \cline{2-5} 
               & \begin{tabular}[c]{@{}c@{}}mlflow\\ (t2.small)\end{tabular}       & -         & 34,54     & 34,54            \\ \cline{2-5} 
               & \begin{tabular}[c]{@{}c@{}}HPMoon nodo\\ compute-0-1\end{tabular} & -         & 3679,15   & 3679,15          \\ \hline
\multirow{2}{*}{Humano}   & \begin{tabular}[c]{@{}c@{}}Ingeniero\\ de software\end{tabular} & 319                 & 8,44/hora          & 2692,36            \\ \cline{2-5} 
               & Doctor                                                            & 15        & 10/hora   & 150              \\ \hline
\textbf{Total} & \textbf{}                                                         & \textbf{} & \textbf{} & \textbf{6605,95} \\ \hline
\end{tabular}
\caption{Tabla de costes}
\label{tab:costes}
\end{table}

Como se puede observar en la Tabla \ref{tab:costes}, el coste aproximado de realización de este trabajo ha sido de 6605,95\euro.

%
%\input{capitulos/05_Analisis}
%
%\input{capitulos/06_Diseno}
%
%\input{capitulos/07_Implementacion}
%
%\input{capitulos/08_Pruebas}
%
%\input{capitulos/09_Conclusiones}
%
%%\chapter{Conclusiones y Trabajos Futuros}
%
%
% \nocite{*}
\bibliography{bibliography/bibliography}%\addcontentsline{toc}{chapter}{Bibliografía}
\bibliographystyle{plain}
%
%\appendix
%\input{apendices/manual_usuario/manual_usuario}
%\input{glosario/entradas_glosario}
% \addcontentsline{toc}{chapter}{Glosario}
% \printglossary
%\chapter*{}
%\thispagestyle{empty}

\end{document}
